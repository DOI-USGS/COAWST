\documentclass[12pt]{book}
\usepackage{html,a4wide}
\usepackage[british]{babel}
\newcommand{\hl}[1]{\htmladdnormallink{{\it #1}}{#1}}
\begin{document}
\pagenumbering{roman}
\pagestyle{empty}

\begin{center}
{\Huge\bf \LaTeX~for dummies}
\end{center}

\cleardoublepage

\noindent
{\Large\bf \LaTeX~for dummies}

\vfill

\begin{table}[htb]
\begin{tabular}{lcl}
by           &:& The SWAN team \\
             & & \\
mail address &:& Delft University of Technology \\
             & & Faculty of Civil Engineering and Geosciences \\
             & & Environmental Fluid Mechanics Section \\
             & & P.O. Box 5048 \\
             & & 2600 GA Delft \\
             & & The Netherlands \\
             & & \\
e-mail       &:& swan-info-citg@tudelft.nl \\
home page    &:& \hl{http://www.swan.tudelft.nl}
\end{tabular}
\end{table}

\vfill

\noindent
Copyright (c) 2012 Delft University of Technology.
\\[2ex]
\noindent
Permission is granted to copy, distribute and/or modify this document
under the terms of the GNU Free Documentation License, Version 1.2
or any later version published by the Free Software Foundation;
with no Invariant Sections, no Front-Cover Texts, and no Back-Cover
Texts. A copy of the license is available at
\hl{http://www.gnu.org/licenses/fdl.html\#TOC1}.

\clearpage
\pagestyle{myheadings}
\newcommand{\chap}[1] % Re-define the chapter command
       {
        \chapter{#1}
        \markboth{\hfill Chapter \thechapter \hfill}{\hfill {#1} \hfill}
       }

\tableofcontents

\chap{What is \LaTeX?} \label{ch1}
\pagenumbering{arabic}

\begin{itemize}
\item \TeX~(pronounced "tech") is a typesetting language.
\item \LaTeX~(pronounced "lay-tech" or "la-tech") is \TeX~with a lot of macros that make it easier.
\item Every mathematician uses it. It's also very widely used among physicists, engineers, and other
      professionals who need to prepare technical documents.
\item \LaTeX~does a good job for documents that need structure, allowing you the
      flexibility to easily modify the text, without worrying about how it's going to
      turn out.
\item To quote Donald Knuth (the inventor of \TeX): "It is intended for the
      creation of beautiful books and especially for books that contain a lot
      of mathematics."
\item There are many different styles of \LaTeX~documents. The most well known are: letter,
      article, report, book and slides.
\item Writing \LaTeX~is just like writing any other language, such as Fortran. You
      write \LaTeX~code in ASCII format in your text editor, then you compile it,
      and look at the output file. In Fortran, your output is an executable program.
      In \LaTeX, your output is a PDF document.
\end{itemize}

\chap{How to obtain \LaTeX?} \label{ch2}

\begin{itemize}
\item You need to get two programs: {\bf MiKTeX} and {\bf TeXnicCenter}\footnote{Note to Linux users:
      \LaTeX~is already available on Linux platforms.}. They are free anyway.
      \begin{itemize}
      \item {\bf MiKTeX} has all the \LaTeX~stuff needed by Windows users.
      \item {\bf TeXnicCenter} is the userfriendly GUI program where you actually type your code.
      \item You need to install {\bf MiKTeX} before you install {\bf TeXnicCenter},
            because {\bf TeXnicCenter} needs to be configured to use the {\bf MiKTeX} files.
      \end{itemize}
\item Here's how to install {\bf MiKTeX}:
      \begin{itemize}
      \item Downloading the wizard will take a while. Live with it.\\Go here
      \hl{http://prdownloads.sourceforge.net/miktex/setup-2.4.1705.exe?download} and download it from a
      mirror.
      \item After downloading the wizard, run it to {\bf download} the {\bf MiKTeX} package. Then, run
      it again to {\bf install} the {\bf MiKTeX} package. Thus, you're going to end up having used the
      wizard twice.
      \end{itemize}
\item Here's how to install {\bf TeXnicCenter}:
      \begin{itemize}
      \item Installation of TeXnicCenter is pretty straightforward. Just make sure you do this after
      installing MikTeX.
      \item Go to\\
      \hl{http://prdownloads.sourceforge.net/texniccenter/TXCSetup\_1Beta6\_31.exe?download}
      and download the setup file from a mirror.
      \end{itemize}
\item Installation reference pages:
      \begin{itemize}
      \item The Art of Problem Solving's \LaTeX~installation guide.\\
            Go to
            \hl{http://www.artofproblemsolving.com/LaTeX/AoPS\_L\_Downloads.php}
      \item {\bf MiKTeX} homepage \hl{http://www.miktex.org/setup.html}
      \item {\bf TeXnicCenter} homepage \hl{http://www.toolscenter.org/}
      \end{itemize}
\end{itemize}

\chap{Start using \LaTeX} \label{ch3}
\begin{itemize}
\item Now you need to know the actual syntax and code to write \LaTeX~and be able produce
      mathematics or physics papers or just adapting the SWAN Technical documentation.
\item The format of a \LaTeX~document is pretty simple:
      \begin{verbatim}
        \documentclass{}

         ... this is called the preamble ....

         \begin{document}

         ... nice text and actual work here ....
         ...  this is called the body  ...

        \end{document}
      \end{verbatim}
\item \LaTeX~allows free form text input. This means that it does not matter how
      many tabs, spaces you have, \LaTeX~has its own idea of what your document will look like.
\item As it goes with teaching yourself any computer language, you learn more by just trying things,
      looking at examples, and searching google for all other answers.
\item Here you find some links to the resources.
\item {\bf Go here first}: \hl{http://www.artofproblemsolving.com/LaTeX/AoPS\_L\_BasicFirst.php}
      It is very good and brief.
\item Short Math Guide for \LaTeX \\
      \hl{http://tex.loria.fr/general/downes-short-math-guide.pdf},\\
      Michael Downes, American Mathematical Society
\item The Not So Short Introduction to \LaTeX \\
      \hl{http://www.ctan.org/tex-archive/info/lshort/english/lshort.pdf}
\item For a good tutorial, go to\\
     \hl{http://www.tug.org.in/tutorials.html}
\item This is a complete course for new \LaTeX~users (note: very large PDF file)\\
     \hl{http://tug.ctan.org/tex-archive/info/beginlatex/beginlatex-3.6.pdf}
\end{itemize}

\chap{Output formats} \label{ch4}

\begin{itemize}
\item Many people output their \TeX~or \LaTeX~into DVI format\footnote{DeVice Independent, a format
     \LaTeX~generates which can later be converted to PDF or Postscript, viewed on screen, printed etc.}.
      But they usually end up
      converting their DVI into Postscript or PDF. You don't need to know anything about DVI.
\item Postscript is a different story. People
      use Postscript because it preserves the look of the document. However,
      one can convert Postscript to PDF and make the file much more portable.
\item With Ghostscript/GSView, you can open Postscript files and just look at them
      like all is good. It is good to be able to do this and to be
      able to convert them to PDF. If it's something worth saving, you'd
      rather have the PDF.
\item You can also convert Postscript to PDF using Adobe Distiller or using {\bf Go2PDF}
      which is free here
      \hl{http://www.go2pdf.com/product.html}.
\item It is also possible to create HTML documents from \LaTeX~files using {\bf \LaTeX2HTML}.
      For more information see \hl{http://www.latex2html.org}.
\end{itemize}

\end{document}
