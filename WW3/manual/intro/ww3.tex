\vssub
\subsection{~The \ws\ Modeling Framework}
\vssub

\ww\ is a community wave modeling framework that includes the latest scientific advancements in the field of 
wind-wave modeling and dynamics.

The core of the framework consists of the \ws\ third-generation wave model, developed at the 
US National Centers for Environmental Prediction (NOAA/NCEP) in the spirit of the WAM model \citep{bk:WAM94}. 
The current framework evolved from earlier WAVEWATCH I \& II model packages \citep{tol:JPO91b, tol:JPO92}, and differs from its predecessors 
in many important points such as governing equations, model structure, numerical methods and physical parameterizations.

\ws\ solves the random phase spectral action density balance equation for wavenumber-direction spectra. The implicit 
assumption of this equation is that properties of medium (water depth and current) as well as the wave field itself 
vary on time and space scales that are much larger than the variation scales of a single wave. The model includes options 
for shallow-water (surf zone) applications, as well as wetting and drying of grid points. Propagation of a wave spectrum 
can be solved using regular (rectilinear or curvilinear) and unstructured (triangular) grids, individually or combined into
multi-grid mosaics.

Source terms for physical processes (source terms) include parameterizations for wave growth due to the actions of wind, exact and parametrized 
forms accounting for nonlinear resonant wave-wave interactions, scattering due to wave-bottom interactions, triad interactions, 
and dissipation due to whitecapping, bottom friction, surf-breaking, and interactions with mud and ice. The model includes several 
alleviation methods for the Garden Sprinkler Effect, and computes other transofrmation processes
such as the effects of surface currents to wind and wave fields, and sub-grid blocking due to unresolved islands. 

Inputs to \ws\ may be provided via external files or via coupling using the OASIS or ESMF/NUOPC frameworks. Input data is
dynamically updated within the wave model driver, and may include ice coverage, mud, current fields, bottom properties for dissipation 
on a moveable bed, and data for assimilation within a data assimilation placeholder module that may be developed by users.

\ws\ is written in ANSI standard FORTRAN 90, fully modular and fully allocatable. The model is set up for traditional one-way nesting, 
and also using a `mosaic' or multiple-grid approach, where an arbitrary number of grids can be considered with full two-way interactions 
between all grids. Individual or multi-grid mosaics can be used as moving frame of reference that allows high-resolution 
modeling of hurricanes away from the coast. 

Wave energy spectra are discretized using a constant directional increment (covering all directions), and a spatially varying wavenumber grid.  
First-, second- and third-order accurate numerical schemes are available to describe wave propagation. Source terms are integrated 
in time using a dynamically adjusted time stepping algorithm, which concentrates computational efforts in conditions with rapid spectral 
changes. \ws\ can optionally be compiled to include shared memory parallelisms using OpenMP compiler directives, 
and/or for a distributed memory environment using the Message Passing Interface.




%, spectral partitioning is now available for post-processing of
%        point output, or for the entire wave model grid using the Vincent and Soille
%        (1991) algorithm (Hanson and Jenssen, 2004; Hanson <I> et al </I>, 2006,
%        2009).  <span style="COLOR:#007f00;"> New in model version 3.14</span> </li>
