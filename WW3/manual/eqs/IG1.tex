\vsssub
\subsubsection{~Second-order spectrum and free infragravity waves} \label{sec:IG1}
\vsssub

\opthead{IG1}{\ws}{F. Ardhuin}

\noindent 
WARNING: {\it A bug has been identified with IG wave sources in unstructured grids. 
A model patch using an older version of the code will be provided shortly.}

The linear dispersion relation used in section \ref{sec:intro} is a good
approximation for most of the wave energy but a significant part of the
spectrum at high frequencies, with typical frequencies above three times the
windsea wave peak \citep[e.g.][]{rep:Lec13}.  In shallow water, another
strongly nonlinear part of the spectrum is found at very low frequencies,
which are called infragravity waves.

In the case of horizontally homogeneous conditions over a flat bottom, both
low and high frequency non-linear components can be estimated from the linear
wave spectrum, using perturbation theory \citep[e.g.][]{art:Has62}. Also, the
non-linear evolution of a homogeneous wave field is better described in terms
of this `linearized spectrum'. It is thus practical to work with this
`linearized spectrum' and convert to the observable spectrum that contains
non-linear components when post-processing the model results. One method to
perform this transformation is a canonical transformation proposed by
\cite{art:Kra94}. The properties of this transformation were further explored
by \cite{art:Jan09} and implemented for post-processing in the ECMWF version
of the WAM model.

The code for the canonical transform written by P. Janssen was interfaced with
\ws.  Using the {\code IG1} switch and setting the parameter {\code IGADDOUTP
= 2} in the {\F SIG1} namelist, this canonical transformed, which conserves
energy, will be used for the output point spectra.  If {\code IGADDOUTP = 1},
then the second-order spectrum is added on top of the model spectrum using the
theory \citep[e.g.][]{art:Has62}. That option does not conserve energy and
is not consistent at high frequency because the quasi-linear term in the
second-order spectrum are ignored \citep{art:Jan09}.

However, when comparing to measurements, one should be aware that different
measuring devices have different responses to the nonlinear part of the
spectrum. In particular, surface-following buoys also linearize the spectrum,
and the second-order pressure field is not related to the second-order
elevation via the relations used for linear waves. The canonical transform is
thus only applicable for wave gauges that measure elevation at a fixed
location.

When the wave field is not homogeneous, the nonlinear properties of the waves
lead to an exchange of energy between different modes. In shallow water this
usually results in the transfer of energy to infragravity waves, that are
released along shorelines and travel as free waves. The {\code IG1} switch
allows the parameterization of that effect with several methods.  These are
very crude parameterizations compared to the full hydrodynamic solution that
would require solving the bispectral evolution across the surf zone at a very
high spatial resolution \citep[e.g.][]{art:HB97}. The default namelist settings
correspond to the parameterization presented by \cite{art:Aea14}, with minor adjustments in version 6.06. 
The power in the definition of $\widehat{E}_{IG}(f)$ was changed from $f^{1.5}$ to $f^{1.0}$, with 
an associated adjuspent of the constant fro 0.015 to 0.013. 

 In practice the free infragravity wave energy is
added via the $S_{ref}$ source term, by setting the {\code SIG1} namelist
{\code IGSOURCE} to 1 or 2.

In the first method, activated with {\code IGSOURCE =1}, the second-order
spectrum is computed using either the Hasselmann perturbation ({\code IGMETHOD
= 1}) or the canonical transform (any other value of {\code IGMETHOD}) as described 
in \cite{art:Jan09}. This
approach may lead to better directional distribution of IG wave energy but it
is still being tested.  The second method, activated with {\code IGSOURCE =
2}, and the free IG spectrum is given by the following expressions,

\begin{eqnarray}
 A_{IG} & =&    H_s T_{m0,-2}^2\label{eq:IGfit0}, \\
\widehat{E}_{IG}(f)& = & 1.2 \alpha_1^2 \frac{k g^2}{c_g 2 \pi f} \frac{(A_{IG}/4)^2}{\Delta_f}  
\left[\min( 1., 0.013\mathrm{Hz}/ f)\right]^{1.0}, \label{eq:IGfit1} \\
 \widehat{E}_{IG}(f,\theta) & = & \widehat{E}_{IG}(f) / (2 \pi ),
\label{eq:fit2} 
\end{eqnarray}

\noindent
where the mean period is defined as $ T_{m0,-2} =\sqrt{m_{-2}/m_{0}}$ with the
moments

\begin{equation}
 m_n= \int_{f_{min}~\mathrm{Hz}}^{0.5~\mathrm{Hz}} E(f) f^n {\mathrm d}f,\label{eq:mn}
\end{equation}

\noindent
and the empirical coefficient $\alpha_1$ is of the order of
$10^{-3}$~s$^{-1}$, and is set by the {\code SIG1} namelist parameter {\code
IGEMPIRICAL}. The minimum frequency $f_{min}$ used to define $ T_{m0,-2}$ is
set by the namelist parameter {\code IGMAXFREQ} and it is also the maximum
frequency of the IG band over which this source of energy is applied.  Also,
in this band the IG energy at the coast can be added on top of pre-existing
energy, or the pre-existing energy can be reset to zero. That latter behavior
is the default and controlled by {\code IGBCOVERWRITE = 1}. For other choices,
({\code IGBCOVERWRITE = 0}), the results are very sensitive to the maximum
shoreline reflection coefficient allowed ({\code REFRMAX} parameter in
namelist {\F REF1}).

Finally, IG energy can also be added for frequencies beyond $f_{min}$, this is the default behavior 
and it is activated by setting {\code IGSWELLMAX = TRUE}.  For that part of the IG wave
field, the IG wave source is now reduced by a factor 4 which is now hard-coded
in {\file w3ref1md.ftn}. This should be adjusted together with the maximum reflection 
which is defined by the {\F REF1} namelist parameter {\code
REFRMAX}. In the present version, the option {\code IGSWELLMAX = TRUE} does 
not work well with unstructured grids. We thus advise to use 
{\code IGSWELLMAX = FALSE} for these grids, this will unfortunately lead 
to a spectral gap between the IG band and the swell-windsea band. 




