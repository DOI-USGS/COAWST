\vsssub
\subsubsection{~Sea-state dependent $\tau$: Reichl et al. 2014} \label{sec:FLD1}
\vsssub

\opthead{FLD1}{\ws}{B. Reichl}

\paragraph{Wind stress according to Reichl et al., 2014}

In \citet{art:Rei14} the total stress is constant in height, but is decomposed into two components as a function of height as:
\begin{equation}
\vec{\tau}=\vec{\tau}_t(z)+\vec{\tau}_f(z),
\end{equation}
where $\tau_t$ is the turbulent stress and is equal to the viscous stress very near the surface.
The wave form stress can be expressed as:
\begin{equation}
\vec{\tau}_f(z)=\rho_w\int^{k=\delta/z}_{k_{min}} \int^\pi_{-\pi}\beta_g(k,\theta) \sigma F(k,\theta) d\theta {\vec{k}} dk,
\end{equation}
that is, the wave form stress at height $z$ is equal to the integration of the wave form stress at the surface for wavenumbers below $k=\delta/z$, where $\delta/k$ is the inner layer height \citep{art:Har04} for waves at a wavenumber $k$.
This expression is derived by assuming that the wave-induced stress is significant from the surface up to the inner layer height, but is negligible further above.
Since at the surface
\begin{equation}
\vec{\tau}=\vec{\tau}_\nu+\vec{\tau}_f(z=0)=\vec{\tau}_\nu+\rho_w\int^{k_{max}}_{k_{min}} \int^\pi_{-\pi}\beta_g(k,\theta) \sigma F(k,\theta) d\theta {\vec{k}} dk,
\end{equation}
the turbulent stress at a height $z$ can be expressed as:
\begin{equation}
\vec{\tau}_t(z)=\vec{\tau}_\nu+\rho_w\int^{k_{max}}_{k=\delta/z} \int^\pi_{-\pi}\beta_g(k,\theta) \sigma F(k,\theta) d\theta {\vec{k}} dk.
\end{equation}

In this model it is assumed that the turbulent stress at the inner layer height $z=\delta/k$ determines the growth rate of waves at wavenumber $k$:
\begin{equation}
\beta_g(k,\theta)=c_\beta \sigma\frac{\left|\tau_t(z=\delta/k)\right|}{\rho_wc^2}\cos^2(\theta-\theta_\tau),
\end{equation}
where $\theta_\tau$ is the direction of the turbulent stress at the inner layer height.
The turbulent stress at the inner layer height is used in place of the total wind stress because longer waves reduce the effective wind forcing on shorter waves (wave sheltering).

The growth rate coefficient $c_\beta$ varies depending on the ratio of the wave phase speed to the local turbulent friction velocity (friction velocity at the inner layer height), $u_\star^l=\sqrt{\tau_t(z=\delta/k)/\rho_a)}$.
\begin{equation}
c_{\beta}=\left\{
\begin{array}{cll} 
25  &: \cos(\theta-\theta_w)>0 \hspace{2mm}&:c/u_{\star}^l<10\\
10+15\cos[\pi(c/u_\star-10)/15]   &:&:10\le c/u_{\star}^l<25 \\
-5  &:&:25\le c/u_{\star}^l\\
-25  &: \cos(\theta-\theta_w)<0 &\\
\end{array}
\right.
\end{equation}

The wind profile is explicitly calculated using the energy conservation constraint in the wave boundary layer.
From the top of the viscous sublayer to the inner layer height of the shortest waves the wind shear is expressed as:
\begin{equation}
\frac{d {\vec{u}}}{\partial z}=\frac{\rho_a}{\kappa z}\left|\frac{ \vec{\tau}_\nu}{\rho_a}\right|^{3/2}\frac{\vec{\tau}_\nu}{\vec{\tau}_\nu\cdot\vec{\tau}_{tot}}\hspace{3mm}{\rm for}\hspace{3mm}z_\nu<z<\delta/k_l.
\end{equation}
Between the inner layer height of the shortest waves and that of the longest waves the wind shear is expressed as:
\begin{equation}
\frac{d {\vec{u}}}{\partial z}=\left[ \frac{\delta}{z^2}\tilde{F}_w\left(k=\frac{\delta}{z}\right)+\frac{\rho_a}{\kappa z}\left| \frac{\vec{\tau}_t(z)}{\rho_a}\right|^{3/2} \right]\times\frac{\vec{\tau}_t(z)}{\vec{\tau}_t(z)\cdot\vec{\tau}_{tot}}\hspace{3mm}{\rm for}\hspace{3mm}\delta/k_l\le z,
\end{equation}
where $\tilde{F}_w(k=\delta/z)$ is the energy uptake by surface waves:
\begin{equation}
\tilde{F}_w(k=\delta/z)=\rho_w\int^{\pi}_{-\pi}\beta_g(k,\theta)gF(k,\theta)k d\theta.
\end{equation}
Finally, above the inner layer height of the longest waves the wave effect is negligible and the wind shear is aligned in the direction of the wind stress:
\begin{equation}
\frac{d{\vec{u}}}{dz}=\frac{u_\star}{\kappa z}\frac{\vec{\tau}_{tot}}{|\vec{\tau}_{tot}|}.
\end{equation}

Note that when using the FLD1 switch, internal variables and output values of the viscous stress, friction velocity, surface roughness length and Charnock parameter are recalculated and overwritten. 
