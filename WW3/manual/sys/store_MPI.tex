\vsssub
\subsubsection{~Distributed memory concepts.} \label{sec:distr}
\vsssub

The general grid structure described in the previous paragraph is used for
both shared and distributed memory versions of the model, with some minor
differences. For the distributed memory version of the model, not all data is
kept at each processor. Instead, each spectrum is kept at a single processor
only. The spectra on the storage grid are distributed over the available
processors with a constant stride. Because only part of the spectra are stored
locally on a given processor, a distinction needs to be made between the above
global sea point counter {\F isea}, and the local sea point counter {\F
jsea}. If the actual number of processors used in the computation is {\F
naproc}, and if {\F iaproc} is the processor number ranging form 1 to {\F
naproc}, these parameters are related in the following way

\vspace{\baselineskip}
\centerline{\F isea = iaproc + (jsea-1) naproc ,}
\centerline{\F jsea = 1 + (isea-1) / naproc ,}
\centerline{\F iaproc = 1 + mod(isea-1,naproc) .}
\vspace{\baselineskip}

\noindent
In model version 3.10, a further refinement was introduced. The actual number
of processors {\F naproc} can be smaller than the total number of processors
used by the program ({\F ntproc}). Processors where {\F naproc} $<$ {\F
iaproc} $\le$ {\F ntproc} are reserved for output processing only.

With this data distribution, source terms and intra-spectral propagation can
be calculated at the each given processor without the need for communication
between processors. For spatial propagation, however, a data transpose is
required where the spectral components {\F(ith,ik)} for all spatial grid
points have to be gathered at a single processor. After propagation has been
performed, the modified data have to be scattered back to their `home'
processor. Individual spectral components are assigned to specific processors
in such a way that the number of partial propagation steps to be performed by
each processor is roughly identical. This makes a good load balance
possible. The actual algorithm can be found in section 4.d of the subroutine
{\F w3init} ({\file w3initmd.ftn}).

The data transpose for the gather operation is implemented in two steps using
the Message Passing Interface (MPI) standard \citep[e.g.][]{bk:GLS97}. First,
values for each spatial grid point for a given spectral bin {\F(ith,ik)} are
gathered in a single target processor in a one-dimensional array {\F
  store(isea)}, which then is converted to the full two-dimensional field of
spectral components. After propagation has been performed, the transpose for
the scatter operation reverses this process, using the same one-dimensional
array {\F store}. Whereas the algorithm for distributing spatial propagation
over individual processors assures a global (per time step) load balance, it
does not assure that communication is synchronized, because not each
calculation at each processor will take the same effort. To avoid that this
results in a load imbalance, non-blocking communication has been
used. Furthermore, the one-dimensional array {\F store(isea)} is replaced by
{\F store(isea,ibuf)}, where the added dimension of the array supplies an
actively managed buffer space (see {\F w3gath} and {\F w3scat} in {\file
  w3wavemd.ftn}). These buffers allow that spare clock cycles as may occur
during communication can be used for calculation, and that hiding of
communication behind calculation will occur if the hardware is capable of
doing this. To avoid problems with incompatibilities between FORTRAN and MPI,
separate gather and scatter data arrays are used.  The buffered data
transposes are graphically depicted in Fig.~\ref{fig:transpose}. More details
can be found in \cite{tol:PACO02}.

% fig:transpose

\setlength{\unitlength}{0.1mm}
\begin{figure}

\begin{picture}(1370,650)(0,-650)

\put(   0, -650){\framebox(1370,650)[c]{}}

\put( 270, -160){\vector(3,-4){150}}
\put( 270, -260){\vector(3,-2){150}}
\put( 270, -360){\vector(1, 0){150}}
\put( 270, -460){\vector(3, 2){150}}
\put( 270, -560){\vector(3, 4){150}}

\put( 820, -360){\vector(1,0){120}}

\put( 420, -380){\vector(-3, 4){150}}
\put( 420, -380){\vector(-3, 2){150}}
\put( 420, -380){\vector(-1, 0){150}}
\put( 420, -380){\vector(-3,-2){150}}
\put( 420, -380){\vector(-3,-4){150}}

\put( 940, -380){\vector(-1,0){120}}

\put(  10,  -80){\makebox(310,80)[c]{processors with}}
\put(  10, -110){\makebox(310,80)[c]{native data.}}
\put(  60, -210){\framebox(210,80)[c]{{\F 1}}}
\put(  60, -310){\framebox(210,80)[c]{{\F 2}}}
\put(  60, -410){\framebox(210,80)[c]{{\F 3}}}
\put(  60, -510){\framebox(210,80)[c]{{\F \ldots}}}
\put(  60, -610){\framebox(210,80)[c]{{\F naproc}}}

\put( 370,  -80){\makebox(500,80)[c]{1-D full-grid array with}}
\put( 370, -120){\makebox(500,80)[c]{single spectral component.}}
\put( 420, -400){\framebox(400,60)[c]{active}}
\put( 420, -460){\dashbox{5}(400,240)[c]{}}
\put( 430, -280){\makebox(380,60)[l]{buffer space}}

\put( 970,  -80){\makebox(270,80)[c]{corresponding}}
\put( 970, -120){\makebox(270,80)[c]{2-D array.}}
\put( 940, -520){\framebox(330,300)[c]{propagate}}

\put( 370, -570){\dashbox{15}(950,400)[c]{}}
\put( 400, -580){\makebox(270,80)[l]{\small at target processor}}

\end{picture}

\caption{Data transpose in distributed memory model version. First, the data is
moved from left to right in the figure during the gather operation. After the
calculation is performed, the data is moved from right to left in the
scatter operation.}
\label{fig:transpose}

\botline
\end{figure}


In principle only the storage array {\F a(ith,ik,jsea)} is influenced by the
data distribution. Input fields, maps and output fields of mean wave
parameters in principle are retained at full resolution at each grid
point. Full maps are available at each processor at each phase of the
calculation. Input and output fields generally contain pertinent data at the
stride {\F naproc} only.

Distributed memory also requires modifications to the I/O. Input files are
read completely by each separate processor. The type of file output is
determined by the I/O type indicator {\F iostyp}.

\begin{center} \begin{tabular}{cl}
{\F iostyp} & implies \\ \hline
  0 & Restart file written from each individual process. \\
  1 & Each file written from assigned process. \\
  2 & Each file written from a single dedicated output process. \\
  3 & Dedicated output processes for each output type.
\end{tabular} \end{center}

\noindent
Note that the restart file is a direct access file, so that each processor can
efficiently gather only the locally stored spectra, without the need of
reading through the entire file. The restart file is either written by each
individual process directly, or all data is funneled through a dedicated
processor. The first method requires a parallel file system, the second method
is generally applicable. 

The present algorithm for data distribution has been chosen for several
reasons. First, it results in an automatic and efficient load balancing with
respect to the (dynamic) integration of source terms, the exclusion of ice
covered grid points, and of intra-spectral propagation. Secondly, the
communication by definition becomes independent of the numerical propagation
scheme, unlike for the more conventional domain decomposition. In the latter
case, only a so-called `halo' of boundary data needs to be converted to
neighboring `blocks' of grid points. The size of the halo depends on the
propagation scheme selected. The main disadvantage of the present data
distribution scheme is that the amount of data to be communicated each time
step is much larger than for a more conventional domain decomposition,
particularly when relatively small numbers of processors are used. On an IBM
RS6000 SP, on which the distributed memory version of \ws\ was tested, the
relatively large amount of communication did not constitute a significant part
of the overall time of computation, and the model shows excellent scaling
behavior for up to O(100) processors \citep{tol:PACO02}.

More recently, hybrid parallelization techniques have been developed using a
combination of a course scale domain decomposition and a local data transpose,
using approaches already available in {\file ww3\_multi}. To accommodate this,
the {file ww3\_gspl(.sh)} tools were introduced in model version
4.10. Although this approach still needs some work with respect to the model
memory footprint in the initialization in {\file ww3\_multi}, initial scaling
results obtained with this approach are encouraging \cite[see][]{tol:MMAB13b}.
